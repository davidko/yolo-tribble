\documentclass[12pt,twoside,letterpaper]{article}
\usepackage{setspace}
\usepackage{fancyhdr}

% the following commands control the margins:
\topmargin=-.5in    % Make letterhead start about 1 inch from top of page 
\textheight=9.0in    % text height can be bigger for a longer letter
\oddsidemargin=0.5in % leftmargin is 1 inch
\evensidemargin=0.5in % leftmargin is 1 inch
\textwidth=6.0in   % textwidth of 6.5in leaves 1 inch for right margin


%%%%%%%%%%%%%%%%%%%%%%%%%%%%%%%%%%%%%%%%%%%%%%%%%%%%%%%%%%%%%%%%%%%%%%%%%
\begin{document}

%%%%%%%%%%%%%%%%%%%%%%%%%%%%%%%%%%%%%%%%%%%%%%%%%%%%%%%%%%%%%%%%%%%%%%
% PageStyle {{{
  \setcounter{page}{1}
  \pagestyle{fancy}
  \fancyhf{} 
  \renewcommand{\headrulewidth}{0.0pt}
  \fancyhead[RE]{\textbf{-\thepage}-}
% }}}
%%%%%%%%%%%%%%%%%%%%%%%%%%%%%%%%%%%%%%%%%%%%%%%%%%%%%%%%%%%%%%%%%%%%%%


{\raggedleft
  Stephen Sodokan Nestinger \\
  September 2009 \\
  Mechanical and Aeronautical Engineering\\
}
\hspace{.5in}

\begin{center}
  A Reconfigurable Cooperative Control System\\
  for Rapid Deployment of Multi-Robot Systems\\
\end{center}

\hspace{.5in}

\centerline{\underline{\textbf{Abstract}}}

\doublespacing
Multi-robot systems have been used in a vast array of fields and are 
  of particular interest in perilous environments.
Utilizing multiple smaller and cheaper robots have many advantages compared to 
  a highly specialized single robot.
Multi-robot systems are fault-tolerant by nature and provide task execution 
  parallelism for faster mission completion.
One of the main issues in multi-robot systems is the lack of a common set of 
  programming and control abstractions and middleware.
Controlling and programming cooperative multi-robot systems is a highly 
  complicated task that requires a flexible and agile control architecture 
  and programming environment that are able to handle the distributed nature of 
  multi-robot systems.
This dissertation studies many different aspects of multi-robot systems.
The major characteristics, different paradigms and programmability of
  multi-robot systems are presented.
The key aspects of cooperative multi-robot systems are discussed along with 
  the different methods in which cooperation is implemented.
The use of mobile agents to provide multi-robot system reconfigurability, 
  reprogrammability, and rapid deployment is introduced.
Several multi-robot system middleware are discussed along with specialized
  middleware for cooperative systems.

\doublespacing
A highly flexible and reconfigurable cooperative robot control platform called
  Mobile-R has been developed in the course of this research.
Mobile-R consists of two modules: the Robot Control System (RCS) and 
  Deployment System (DS).
Mobile-R is a highly extensible platform that follows the multi-agent 
  paradigm.
It allows for the implementation of architectures popularly used in the 
  different multi-robot paradigms and is based on widely accepted standards
  for multi-agent interaction allowing for interoperability with other 
  multi-agent systems. 
Mobile-R is built upon Mobile-C, an IEEE  Foundation for Intelligent 
  Physical Agents standards compliant mobile agent system. 
The innate mobility characteristic of mobile agents provides an invariant 
  execution of control code over disparate hosts and overall system fault 
  tolerance.
The system has been validated through multiple experiments presented in the
  dissertation.
The simulated application of Mobile-R to tier-scalable planetary reconnaissance 
  demonstrates the feasibility and applicability of the system to various 
  multi-robot scenarios.

\end{document}
