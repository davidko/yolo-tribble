%%%%%%%%%%%%%%%%%%%%%%%%%%%%%%%%%%%%%%%%%%%%%%%%%%%%%%%%%%%%%%%%%%%%%%%%%%%%%%%%
\chapter{Conclusions and Future Work}

  %%%%%%%%%%%%%%%%%%%%%%%%%%%%%%%%%%%%%%%%%%%%%%%%%%%%%%%%%%%%%%%%%%%%%%%%%%%%%%
  \section{Conclusions}
  %%% {{{
    This research is motivated by the need for a versatile multi-robot
      system middleware that provides the reduced expenditure overhead,
      robustness to failures, ease of use, generality, and flexible
      extendability.
    The research mainly focused on the design and implementation of a
      general, versatile and highly flexible multi-robot control
      middleware, encompassing a robot control system and deployment 
      system.
    The major contributions of this dissertation are as follows:
    \begin{itemize}
      \item \textbf{Developed a versatile, highly flexible and generalized
        middleware system - Mobile-R.} 
        Mobile-R is a highly extensible and reconfigurable middleware system
          that follows the multi-agent standards of the Foundation for 
          Intelligent Physical Agents (FIPA).
        It allows for the implementation of architectures popularly used in the 
          different multi-robot paradigms.
        Mobile-R is based on widely accepted standards for multi-agent 
          interaction allowing for interoperability with other multi-agent systems.
        Using mobile agents provides innate fault-tolerance by using the mobile
          agents' ability to migrate to different hosts.
        Mobile-R contains all of the salient features of an ideal robot 
          middleware.
        The system is based on the standard C/C++ programming language and 
          provides a simple interface for extending functionality to reduce 
          the computational and programing efforts and it is easy to use.
%        Mobile-R is highly reconfigurable allowing the implementation of
%          almost all existing multi-robot architectures with ease.
        Since Mobile-R is based on C/C++, the system can easily be 
          extended and existing C/C++ based software can quickly be integrated
          into the system.
        Mobile agents provide a mechanism for rapid reprogrammability of a
          system and migration capabilities between virtual robots in a 
          simulated environment and and real-world physical systems.
        
      \item \textbf{Explored the efficiency of a mobile agent-based multi robot 
                    middleware.}
         Using Mobile-C as the mobile agency with C/C++ mobile agents has many 
           distinctive advantages.
         In general, robot manufacturers provide system libraries to control
           their robots in C/C++.
         Using a system that shares the same programming language provides easy
           integration of the software with the low-level drivers.
         The user can use C to easily interface with low-level hardware while using
           C++ to create a object oriented high-level interface.

      \item \textbf{Extended the IEEE FIPA standards to support physical services.}
        FIPA specifications mainly allow for inter-agent software shared 
          services.
        However, when dealing with physical robot systems, it may be possible
          for a robot system to share physical services such as having an
          installed manipulator.
        The service specifications and sharing protocol for physical 
          capabilities have been developed on top of the preliminary FIPA 
          service discovery agent specifications.

      \item \textbf{Extended the capabilities of the Mobile-C mobile agent system.}
        The Mobile-C mobile agent system was extended to include a service
          discovery agent that provides remote service discovery
          across the domain.
        The mobile agent system was also enhanced with a FIPA communication
          modules that off loads all of the communication processing from
          the agent to the agency.
        It also significantly reduces the amount of agent code written by a 
          user in order to use the FIPA ACL communication language.

      \item \textbf{Developed an accompanying extendible deployment system.}
        A deployment system was designed and partially developed to provide 
          rapid deployment of mission-based multi-robot systems.
        The deployment system allows for custom defined architectural 
          components and reconfiguration or installation of the necessary
          mobile agent on individual robots.

      \item \textbf{Provided a mobile agent-based multi-robot architecture
                    for planetary reconnaissance and structure deployment.}
        A cooperative mission-based tier-scalable planetary reconnaissance
          and structure deployment system has been designed and partially
          implemented using Mobile-R.
        The use of Mobile-R allows for automatic task delegation and
          decomposition with collaborative capabilities.



    \end{itemize}
  %%% }}}
  %%%%%%%%%%%%%%%%%%%%%%%%%%%%%%%%%%%%%%%%%%%%%%%%%%%%%%%%%%%%%%%%%%%%%%%%%%%%%%

  %%%%%%%%%%%%%%%%%%%%%%%%%%%%%%%%%%%%%%%%%%%%%%%%%%%%%%%%%%%%%%%%%%%%%%%%%%%%%%
  \section{Future Work}
  %%% {{{

    As previously described, this research has completed with the design of a 
      general, versatile and flexible multi-robot control platform, Mobile-R,
      including a robot control system and a deployment system.
    The main components of the system have been implemented, validated in both 
      physical and virtual robot systems, and simulated in a planetary
      reconnaissance system.
    However, multi-robot systems encompass a wide variety of research and 
      several issues still remain to be studied.

    The Mobile-R deployment system still needs to be expanded to include
      quick addition of user definable system components.
    Currently, Mobile-R takes in the user input concerning group assignment
      and choosing a specific architectural implementation for specific missions.
    However, it would be more beneficial to users to have the automatically 
      decide which architectural implementation would be more appropriate and 
      which robots should be grouped together.
    The system should also be expanded to include the control of other widely 
      used robots.

    Mobile-R provides the necessary foundation for adding a mission 
      learning mechanism in which robots returning from a mission or a task can
      provide feedback to the deployment system.
    The deployment system can learn individual robot characteristics along with
      groups arrangements and integrate in-field learned behaviors.

    Although Mobile-R provides some advanced functionality including 
      the integration of a genetic algorithms, vision, and an artificial neural 
      network library, more advanced features should be included such as
      simultaneous localization and mapping with distributed algorithms,
      multiple filter implementations and other useful multi-robot 
      functionality.

    The system validation and testing in Chapter \ref{sec:validation} 
      demonstrates the vast capabilities of Mobile-R.
    The next stage would be a comprehensive performance comparison with other
      middleware, and between distributed and centralized control scheme.
    For example, it would be interesting to figure out at what point, 
      how many robots, would it be best to utilize a distributed control 
      scheme instead of a centralized one.

    Currently, the deployment system robot-to-task selection criteria is only
      based on whether or not the robot can fulfill the task. 
    However, full optimization of robot selection should include the dynamic
      characteristics of the robot as well.
    Therefore, although any fast multi-wheeled robot is capable of traversing 
      over bumpy terrain and rapidly pursuing an evader, those with nominal
      suspension system would be better.
    Task execution time should also be taken into consideration.
    The robot-to-task selection criteria should be improved to include 
      individual robot dynamics and a task execution time estimator.

    The developed deployment system provides based functionality through a
      prompt based interface.
    User are more receptive to GUI-based systems as compared to prompt-based
      applications.
    Therefore, the menu-driven deployment system should be re-implemented
      with a fully functional GUI user interface.
  %%% }}}
  %%%%%%%%%%%%%%%%%%%%%%%%%%%%%%%%%%%%%%%%%%%%%%%%%%%%%%%%%%%%%%%%%%%%%%%%%%%%%%
 
%%%%%%%%%%%%%%%%%%%%%%%%%%%%%%%%%%%%%%%%%%%%%%%%%%%%%%%%%%%%%%%%%%%%%%%%%%%%%%%%
