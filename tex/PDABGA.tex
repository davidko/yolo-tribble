\chapter{The Persistent Distributed Agent-Based Genetic Algorithm}
  The Persistent Distributed Agent-Based Genetic Algorithm (PDABGA) 
    is an island-type distributed genetic algorithm (GA). 
  The general goal for the new algorithm are to optimize extremely
    complex problems utilizing a distributed network of computing hosts
    over an indefinite period of time.
  These goals are accomplished by utilizing mobile-agent technology in the 
    design of the algorithm.

  The population of the of the GA is composed of autonomous mobile agents.
  Each agent in the agent population is responsible for its own fitness
    calculation, mate selection, and reproduction.
  PDABGA is
    comprised of a handful of individual components, including a Mobile-C
    agency, various Mobile-C agents, a simulation framework, and a
    newscast network binding the distributed agencies together, as summarized
    in Figure \ref{fig:pdabga_architecture}.
  The following sections will disseminate each of the components in detail.

  \section{Mobile-C: A C/C++ Mobile Agent Framework}
    Mobile-C was first conceived and prototyped as a standalone application which
      processed agents written in C embedded in XML \cite{chen2005}. 
    During the course of this research, it has evolved to become an embeddable,
      fast, and stable agent platform. 
    Mobile-C has already been used in several robotic systems, such as robotic
      workcells \cite{Nestinge2010b}. 
    In the robotic workcells, the agents take advantage of agent
      synchronization methods provided by Mobile-C to perform a coordinated task.
    Mobile-C has also been used on mobile robots performing distributed vision
      sensor fusion \cite{Nestinge2010}. 
    By utilizing mobile agents, image processing is done in situ on the robots,
      thereby saving network bandwidth and energy.

  \section{Agency Design}
    \subsection{Initialization}
      The Agency performs a variety of steps upon startup.


  \section{Mobile Agent Design}
    \subsection{The GA Agent}
    Each GA Agent within the PDABGA is analagous to a living entity in the real world.
    The agents carry genetic material, can travel to new locations, look for
      mates, have children, and susceptible to death depending on their fitness.
    The agents are implemented as a state machine capable of carrying multiple
      conversations simultaneously with other agents and agencies. 
    The state machine is summarized in Figure \ref{fig:gaAgentStateMachine}.
    Note that the state machine is implemented on a per-conversation basis, such that
      multiple conversations can be maintained by a single agent simultaneously. 

    \subsection{The Master Coordinator Agent}

    \subsection{The Cull Agent}

    \subsection{The Newscast Agent}


